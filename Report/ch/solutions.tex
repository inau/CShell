\chapter{Solution}
This chapter is about the different approaches we used to solve the different requirements[See Chapter 2].
\subsection{R1: Running independent}
The BOSC skeleton more or less already provided this feature.

\subsection{R2: Displaying the hostname}
When solving this requirement we used our knowledge of the linux kernel and the way you access system information in the filesystem. The information we need is stored in the file system under the path /proc/sys/kernel/hostname.

We used two methods to get the hostname.
\begin{itemize}
\item fopen(const char *path, const char *mode) which opens a filestream
\item fgets(char *s, int size, FILE *stream) which we used to read out a string of the open file
\end{itemize}
We removed the last char to avoid the nextline character which we didn't need.

\subsection{R3: Implementing use of standard commands}
To support the ls, wc and similar features we programmatically checked the environmental path for existing commands, if none were found we threw an error message.

To achieve this we used the function getenv(constant char *name), with the input "PATH".

\subsection{R4: Implementing Backgrounding}
To achieve the backgrounding feature we needed the fork() method, additionally we added a statement in the execute command loop[See appendix \ref{app:B}, L??].\\
We checked whether the background property of the single command struct equalled 0, if this was the case we should not wait for the current process.

\subsection{R5: Implementing Redirection}
We need following methods to achieve the redirection.
\begin{itemize}
\item int close(int fd)
\item int dup(int oldfd)
\item FILE *fopen(const char *path, const char *mode)
\end{itemize}
The function close(int fd) closes the given file descriptor if successful. With the knowledge that dup() always uses the lowest available filedescriptor and duplicates the given one, we know that using close followed by dup would replace that filedescripter.
Additional usefull information is that std\_in equals 0, std\_out equals 1 and std\_err equals 2.\\
So to  redirect input we close(0), open a filestream, dup the filestreams descriptor and close(1) instead to redirect output.

\subsection{R6: Implementing Piping}
Our piping solution ended up not working. We made several attempts, which is outlined in CSHELL.c in the pipecmd method.
It basically works by getting a command and checking if the command is the first, a middle, or the last command.

We attempted with a loop, where we have two pipes which we alternate between while processing the command chain.

When its the first command it knows its supposed to pipe the given std\_out to the next commands std\_in.

In case its a 'middle' command we know that both input and output should be piped - and this is where we need two pipes. The first pipe is a handle to the previous commandos output where as the second pipe is set to the output.

In the next iteration these handles are turned and the one used to pipe output, is used to provide input for the next command.
Its important to close all ends of a pipe, else deadlocks can occur since no EOF will be sent, and thus never terminate.

In case of the command not having a 'next' it must either be a single command or the last command, hence we change the stdout back to the shellCMD, in order to print the results.
  
\subsection{R7: Implementing exit}
We added definitions in the beginning, similar to the ones in the parser. The difference was that we used char* instead of char.
\begin{itemize}
\item EXIT	("exit")
\item CD      ("cd")
\item isexit(c) (strcmp(c,EXIT) == 0)
\item iscd(c) (strcmp(c,CD) == 0)
\item isspec(c) (isexit(c) || iscd(c))
\end{itemize}


\subsection{R8: Implementing break}
\begin{itemize}
\item signal(SIGINT, catch\_int)
\end{itemize}

The key combination Ctrl-C raises a signal with a constant SIGINT.
In our main method we register a signal handler to listen for SIGINT, and then call the catch\_int method.

The catch\_int method then calls the killProcessId method, which kills the process that is stored in a global variable, as long as the process is not 0.

When we start a foreground process the global process id, is set to that process id, and thus ctrl+c enables us to terminate that process.

If the process should finish by itself, we set the global process id variable back to 0, thus disallowing ctrl+c to break anything.

We only need a single global variable, as it is not possible to run more than a single foreground process simultaneously. 

\subsection{F1: Change directory function}
To support this feature we need following methods:
\begin{itemize}
\item \begin{verbatim}chdir(char* path)
\end{verbatim}
\item \begin{verbatim}getcwd(char*path, t_size t)
\end{verbatim}
\end{itemize}
The chdir() method, changes the current running programs working directory. This means that programs executed, after chdir() has altered the 'current working directory' will be ran within aforementioned directory, ie. running ls will return the contents of that specific directory.\\

The getcwd() method is a getter for accessing the current working directory.

\subsection{F2: Coloring of ls output}
Due to time restrictions, we have not finished this feature - but nevertheless some aspects of our  shell are quite colorfull.