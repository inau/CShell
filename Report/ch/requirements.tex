\chapter{Requirement Specification}
This chapter describes the requirements from which we have based our CShell.

\section{Mandatory Requirements}
Described by the assignment.

\begin{itemize}[leftmargin=25mm]

\item[R1:] The shell should run independently.
\item[R2:] The shell should show the name of the host it runs on
\item[R3:] A user should to be able to input commands such as ls, cat or wc. If unknown commands are inputted, the shell should return a error message.
\item[R4:] The shell should support multiple background operations by use of the '\&' operator.
\item[R5:] The shell should support redirection of standard input and output. Ie. to a file.
\item[R6:] The shell should support piping. Ie. calling: \begin{verbatim}
	ls | wc -w should print the # of files.
\end{verbatim}
\item[R7:] The keyword exit should be implemented, terminating the shell.
\item[R8:] The command [crtl] + [c] should break running commands, but not the shell.

\end{itemize}

\section{Additional Requirements}
Additional features we would like to implement

\begin{itemize}[leftmargin=25mm]

\item[F1:] The shell should make a pacman like figure eating '-' when loading.
 C- - - - -

\end{itemize}